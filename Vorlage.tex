%hier werden die Haupteigenschaften des Dokumentes parametriert
%Die Schriftgröße ist 12pt, die Papiergröße A4, die Dokumentenklasse ist scrartcl (eine KOMA-Skript-Klasse)
%die Option listof=entryprefix sorgt dafür, dass sich die Bezeichnungen der Abbildungen und Tabellen ändern lassen (Zeile 13,14)
%die Option captions=tableheading verändert die Formattierung der caption von Tabellen, da diese über der Tabelle gesetzt wird
\documentclass[
	12pt, 
	a4paper, 
	listof=entryprefix,
	captions=tableheading
	]{scrartcl}
\pdfminorversion=7

\providecaptionname{ngerman}{\listoflofentryname}{Abb.}			%setzt den Namen der Abbildungen
\providecaptionname{ngerman}{\listoflotentryname}{Tabelle}		%setzt den Namen der Tabellen
\usepackage{caption}[center]										%zentriert den Text einer caption
%Bearbeitung der Tabellencaption:
\captionsetup[table]{labelsep=newline,format=plain,width=0.5\textwidth,  justification=centering, position=above}		


\usepackage[ngerman]{babel} 								%neue Rechtschreibung und Silbentrennung
\usepackage[utf8]{inputenc}								%ermöglicht Umlaute
\usepackage[T1]{fontenc}									%Umlaute gemeinsam mit inputenc (8-bit Kodierung)
\usepackage{mathptmx}
\DeclareSymbolFont{Symbols}{OMS}{zplm}{m}{n}				%Times New Roman
\DeclareMathSymbol{\Infty}{\mathord}{Symbols}{"31}		%Unendlichkeitssymbol hinzufügen
\setkomafont{disposition}{\normalfont\bfseries}			%Stil in Matheumgebung

\usepackage{color}											%Text ist färbbar
\usepackage{microtype}	    								%Blocksatz
\usepackage{setspace} 										%Einstellung Zeilenabstand
\onehalfspacing 											%1.5 Zeilenabstand
						
\def\theequation{\thesection.\arabic{equation}}			%Formelnummerierung entsprechend dem Kapitel
\usepackage{amsmath}										%für mathematische Formeln
\usepackage{physics}										%Ableitungsabkürzung
\usepackage{amsfonts}										%weitere mathematische Symbole
\usepackage{amssymb}										%Symbole wie Pfeile, etc.
\usepackage{tikz}											%Diagramme und Zeichnungen
\usepackage{tikz-3dplot}									%3D-Plots
\usetikzlibrary{matrix,calc,patterns,angles,quotes}
\usepackage{pgfplots}										%Diagramme
\usepackage{circuitikz}									%Schaltbilder
\usepackage{makecell}										%Zeilenumbruch in Tabelle
\usepackage{longtable}										%Seitenübergreifende Tabelle
\usepackage[colorlinks=true,								%alle Links schwarz
        linkcolor=black,
        citecolor=black,
        filecolor=black,
        pagecolor=black,
        urlcolor=black,
        bookmarks=true,
        bookmarksopen=true,
        bookmarksopenlevel=3,
        plainpages=false,
        pdfpagelabels=true]{hyperref}
\usepackage{romannum}
\addto\extrasngerman{
\renewcommand{\figureautorefname}{Abb.}					%Referenzbezeichnung für Abbildungen
\renewcommand{\tableautorefname}{Tabelle}					%Referenzbezeichnung für Tabellen
\renewcommand{\equationautorefname}{}						%keine Referenzbezeichnung für Gleichungen
}

\usepackage[
	left=4cm,
	right=2cm,
	top=2cm,
	bottom=2cm
	]{geometry}												%Seitenränder
	
\usepackage{scrlayer-scrpage}								%Zentrierung der Seitenzahl
\pagestyle{scrheadings}
\clearpairofpagestyles										%nothing in the headline
\chead{\pagemark}											%pagemark centered in headline
	
\usepackage{tocbasic}										%Inhaltsverzeichnis
\usepackage[nohyperlinks]{acronym}							%Abkürzungen
\usepackage[style=ieee, sorting=none]{biblatex}  			%Zitierstil
%\addbibresource{Bachelorarbeit.bib}						%Literaturdatei, am besten man nutzt Zotero oder andere Literaturverwaltungsprogramme

\usepackage{pdfpages}
\begin{document}
\renewcommand\tablename{TABELLE}							%Tabellenüberschrift
\renewcommand\thetable{\Roman{table}}						%Beschriftungsart Tabelle
\renewcommand\figurename{Abb.}								%Abbildungsüberschrift
\newcommand{\autoeqref}[1]{\hyperref[#1]{\equationautorefname~(\ref*{#1})}} %Gleichungsreferenzierung


\makeatletter
	\renewcommand\@dotsep{10000}
\makeatother

%\includepdf[pages=1]{Titelblatt_Bachelorarbeit.pdf}  	%Titelblatt
%\includepdf[pages=1]{Thema_Bachelorarbeit.pdf}  			%Thema
%\includepdf[pages=1]{Einreichung_Thesis}					%Einreichungsformblatt
\clearpairofpagestyles
\begin{flushleft}
\section*{Autorenreferat}
\end{flushleft}
\newpage
\pagenumbering{arabic}
\tableofcontents
\newpage

\addcontentsline{toc}{section}{Abkürzungsverzeichnis}
\section*{Abkürzungsverzeichnis}
\begin{acronym}
\acro{BLDC}[BLDC-Motor]{bürstenloser Gleichstrommotor}
\acroplural{BLDC}[BLDC-Motoren]{bürstenlose Gleichstrommotoren}
\end{acronym}
\newpage

\section*{Verzeichnis mathematischer Symbole}
\addcontentsline{toc}{section}{Verzeichnis mathematischer Symbole}
\begin{longtable}{cp{0.7\textwidth}}
  $F$ & Kraft\\
  $I$ & Strom
\end{longtable}
\setcounter{table}{0}
\newpage

\addcontentsline{toc}{section}{Tabellenverzeichnis}
\listoftables
\newpage

\addcontentsline{toc}{section}{Abbildungsverzeichnis}
\listoffigures
\newpage
\chead{\pagemark}
\thispagestyle{empty}
\section{Einleitung}
Dieses Dokument kann als Vorlage für wissenschaftliche Arbeiten an der Berufsakademie Bautzen, in technischen Studiengängen verwendet werden. Es gibt keine Gewähr auf Richtigkeit oder Vollständigkeit der Formatierung!\\
Im Folgenden sind ein paar grundlegende Beispiele gegeben die als Anhaltspunkt dienen sollen.
\section{Grundlegende Elemente}
\subsection{Bild einfügen mit Gleitumgebung}
Gleitumgebungen ermöglichen das dynamische Einbinden einer Abbildung. Hierbei wird das Bild dort positioniert wo es passt.
\begin{figure}[h]							%[h]->wird wenn möglich an dieser Stelle platziert
\centering
\includegraphics[scale=0.5]{Testbild.jpg}
\caption{EINGEFÜGTES BILD}
\label{fig:Testbild}
\end{figure}
\subsection{Bilder zeichnen mit tikzpicture und Nutzung von minipages}
\begin{minipage}{0.45\textwidth}
\begin{tikzpicture}
\coordinate (a) at (7,3.5);
\coordinate (b) at (3.5,3.5);
\coordinate (c) at (5,6.5);
\coordinate (d) at (5.5,6);
	\draw[thick,->](0,3.5)--(7,3.5)node[anchor=west]{$\alpha$};
	\draw[thick,->](3.5,0)--(3.5,7)node[anchor=west]{$j\beta$};
	\draw[thick,->, color=blue](2,0.5)--(5,6.5)node[anchor=west]{$\mathrm{q}$};
	\draw[thick,->, color=blue](6.5,2)--(0.5,5)node[anchor=south]{$\mathrm{d}$};
	\draw[thick,->, color=green] (1.5,1)--(5.5,6)node[anchor=north west]{$\gamma$};
	\draw[thick,->, color=green] (6,1.5)--(1,5.5)node[anchor=south]{$\delta$};
	\pic [pic text={$\theta$}, draw, thick, ->,angle radius = 1cm, angle eccentricity=0.7] {angle = a--b--c};
	\pic [pic text={$\hat{\theta}$}, draw, thick, ->, angle radius = 2.5cm, angle eccentricity = 0.7]{angle = a--b--d};
	\pic [pic text={$\overline{\theta}$}, draw, thick, ->, angle radius = 2.8cm, angle eccentricity = 0.85]{angle = d--b--c};
\end{tikzpicture}
\captionof{figure}{TIKZPICTURE}
\label{fig:tikzpicture}
\end{minipage}
\hfill
\begin{minipage}{0.4\textwidth}
\centering
\textbf{\{Hier könnte dein Text stehen\}}
\end{minipage}
\subsection{Einbinden von PDFs als Bild}
\begin{center}
	\includegraphics[scale=0.2]{Testbild.pdf}
	\captionof{figure}{PDF ALS BILD}
\end{center}
\subsection{Einbinden von csv-daten und erstellen von Diagrammen}
\begin{center}
	\begin{tikzpicture}
		\begin{axis}[width=\textwidth,height=0.2\textheight,
				title={Diagrammtitel},
				xtick={0, 100, 200},
				xlabel={x-Achse},
				extra y ticks={},
				ylabel={y-Achse},
				]
			\addplot table[x=Zeit, y=nichtZeit, col sep = semicolon]{test.csv};
		\end{axis}
	\end{tikzpicture}
\captionof{figure}{DIAGRAMM AUS CSV-DATEI}
\label{fig:CSVDiagramm}
\end{center}
\subsection{Schaltplan mit circuitikz}
\begin{center}
\begin{circuitikz}[european]
\draw(0,0)to [vsource, v=$u_{St}(t)$]++(0,2);
\draw(0,2)to [R, l_=$R_{St}$]++(4,0);
\draw(4,2)to [L, l_=$L_{St}$]++(2,0);
\draw(8,2)to [vsource, v=$u_{P}(t)$]++(-2,0);
\draw(8,2)--++(0,-2)--++(-8,0);
\end{circuitikz}
\captionof{figure}{BEISPIELSCHALTUNG MIT CIRCUITIKZ}
\label{fig:Beispielschaltung}
\end{center}
\subsection{Darstellung von Gleichungen}
In der \textit{align} Umgebung werden Gleichungen fortlaufend dargestellt, es ist möglich diese am = oder anderen Stellen gleich auszurichten.
\begin{align}
\intertext{Kommutativgesetz:}
	c &= a+b			\\
	  &= b+a			%Das & vor dem = bewirkt das untereinanderschreiben der Gleichungen
\end{align}
In der \textit{equation} Umgebung können nur einzelne Gleichungen dargestellt werden, die Ausrichtung erfolgt immer zentriert.\\
Kommutativgesetz:
\begin{equation}
c = a+b
\end{equation}
\begin{equation}
=b+a
\end{equation}
\subsection{Erstellen von Tabellen}
\begin{center}
\captionof{table}{Tabellenerstellung}				%die caption muss über der Tabelle gesetzt werden, damit diese als Tabellenüberschrift erscheint!
\label{tab:Tabelle1}
\begin{tabular}{|c|c|}
\hline
\makecell{Titel1}&\makecell{Titel2}\\
\hline
\makecell{Wert1}&\makecell{Wert2}\\
\hline
\end{tabular}
\end{center}
\newpage
\thispagestyle{empty}
\newpage
\addcontentsline{toc}{section}{Literaturverzeichnis}
\printbibliography  
\newpage
\section*{Anlagenverzeichnis}
\begin{itemize}
	\item Anlage I  	- Lastenheft
\end{itemize}
\newpage
\pagestyle{empty}
\newgeometry{left=2cm,
	right=2cm,
	top=2cm,
	bottom=2cm}
\null \vfill
  \begin{center}
 	\Huge {Anlagen}\\
 	\vspace*{0.5cm}
 	zur\\
 	\textbf{Projekt-, Seminar-, Bachelorarbeit}\\
 	\vspace*{0.5cm}
 	---Thema---
\\
 	\vspace*{0.5cm}
 	eingereicht von\\
 	\vspace*{0.5cm}
 	Max Mustermann\\
 	Seminargruppe 1XXXX-1\\
 	Matrikelnr. 100xxxx
   \end{center}
 \null \vfill
\newpage
\newgeometry{left=4cm,
	right=2cm,
	top=2cm,
	bottom=2cm}
\section*{Selbstständigkeitserklärung}
Ich erkläre an Eides statt, dass ich die vorliegende Arbeit (entsprechend der genannten Verantwortlichkeit) selbstständig und nur unter Verwendung der angegebenen Quellen und
Hilfsmittel angefertigt habe.
Die Zustimmung der Firma zur Verwendung betrieblicher Unterlagen habe ich eingeholt.
Die Arbeit wurde bisher in gleicher oder ähnlicher Form weder veröffentlicht noch einer anderen
Prüfungsbehörde vorgelegt.
\newline
\newline
\newline
\newline

\begin{flushleft}
\begin{tikzpicture}
	\draw (0,0)--(15,0);
\end{tikzpicture}

Ort, Abgabetermin \hfill Unterschrift des Verfassers
\end{flushleft}
\end{document}
